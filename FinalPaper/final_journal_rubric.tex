\title{Rubric for the Final Journal Compilation, INTD255}
\author{Dr. Jordan Hanson - Whittier College Dept. of Physics and Astronomy}
\date{\today}
\documentclass[10pt]{article}
\usepackage[a4paper, total={18cm, 27cm}]{geometry}
\usepackage{outlines}
\usepackage[sfdefault]{FiraSans}
\usepackage{hyperref}

\begin{document}
\maketitle

\section{The Goal}
The goal of this assignment is to answer one simple question: in your life after Whittier College, \textit{what will you explore?}  Often people move to the next phase of their life thinking it is the right path.  We land our first jobs, we think about where to live, and we settle down.  Very little thought is given to exploration, either because people have the misguided notion in their heads that there is not much left to explore that is new in the world, or that it is too hard or time consuming.  Some people know it's not for them.

And yet...there \textit{is} so much to explore out in the world.  I hope it has become clear throughout this course that the journal assignments were as much about \textit{self exploration} as they were \textit{geographical} exploration.  Further, exploration can take many forms.  Spiritual exploration can lead one to a better study of the soul.  Romantic exploration can lead to a deeper and more self-less relationship with a partner.  Academic exploration can lead to new discoveries, even if they are not far off in the wilderness.  Finally, physical exploration is still done in so many sites around the world, even in an era when satellite imagery seems to have illuminated the globe.

\section{The Structure of the Writing}
The basic structure of this assignment is to concatenate and blend your journal entries into one long piece of writing.  The length should be around 10 pages double-spaced, but the length should not be important.  \textbf{The key is that you must incorporate content from your all your journal entries.}  For the style and tone of the writing, I want you to take as a model the writing of Barry Lopez in \textit{Horizons.}  We read his longer essay about traveling through Antarctica.  Much of his writing is about the concept of \textit{place.}  That is, on the surface he seems to be describing a place for the reader.  But underneath the surface he seems to be choosing details to illustrate deeper meanings he finds in place he travels.  Thus, I want you to reflect on what places or areas of your life you will explore once you leave Whittier College, and build this reflection from your journals.

\section{List of Journals We've Completed, How Each Relates to the Goal}
\begin{enumerate}
\item Survival situations, week 1.  Places worth exploring always carry some risk.  How does the area you wish to explore involve planning for survival?
\item Different, week 2.  Inevitably, exploring anything takes you to a different place.  How does your ability to experience different people or places play a role in exploration?
\item Diet, week 3.  Learning to care for yourself is a lifelong skill of great value.  How did your writing on either diet, or counting steps inform how you can explore?
\item Pushing yourself, week 4.  In what ways will your area of exploration force you to push yourself either mentally or physically?
\item Over the seas, week 6.  Where in the world will your area of exploration take you?  How will you get there?
\item Small, week 8.  In any journey, people always encounter moments, ideas, or places that are larger than themselves.  Knowing what you wrote about feeling small, do you think you would encounter the feeling of smallness in your area of exploration?
\item The Goal of My Lfe, week 9.  How did you determine the goal of your life?  Roald Amundsen knew the goal of his life would always be to land at the North Pole.  He took the next best thing, the South Pole, and he was arguably the first person to reach the North Pole with navigational precision and record keeping.  Ultimately, a rescue mission for another party returning from the North Pole cost him his life.  But throughout his life that goal of reaching the Poles always guided each small decision he made.  How does your journal writing on your life's goal reveal the effect on your daily decisions by your goal?
\item Lost, week 11.  In working toward your goal, your area to explore, there will be inevitable losses.  You will have to backtrack, fix mistakes, find your way forward.  How does your journal writing teach you what is necessary for finding your way back from being lost?
\item Leadership, week 13.  Ultimately, you will be the leader of your own exploration.  No one will do it for you.  How does your journal writing show you what is necessary to lead yourself down the right path in your exploration?
\end{enumerate}

\section{The Specifics}
The writing should be your own style, voice, and structured as you wish.  However, there are a few guiderails:
\begin{enumerate}
\item About 10 pages, double-spaced.
\item Incorporate writing from each of the 9 journal entries we've done.
\item Answer the question: what will I explore in my life?
\item The essay should be about one topic, not several.
\item The form and organization of the essay \textit{will not be graded.}  Examples include sentences per paragraph, diction, citation format, etc.
\item 25 percent of the grade will be inclusion of content from each journal entry.
\item 25 percent of the grade will be clearly identifying your area of exploration.
\item 25 percent of the grade will be keeping the thread of the essay intact, meaning the reader can follow the reasoning until the end.
\item 25 percent of the grade will be for including the idea of \textit{place}, meaning that describing a place or area you will explore on a surface level is augmented by reflecting on the deeper meaning of that place.
\end{enumerate}

\section{Grading}
This piece of writing is for you.  It is for the development of yourself, and therefore it will not be graded harshly.  I want you to open up and explore in your mind through the writing.  Give yourself permission to be creative.  However, pleace follow the general rules in the prior section.
\end{document}
