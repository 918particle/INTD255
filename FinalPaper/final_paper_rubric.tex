\title{Rubric for the Final Science Paper, INTD255}
\author{Dr. Jordan Hanson - Whittier College Dept. of Physics and Astronomy}
\date{\today}
\documentclass[10pt]{article}
\usepackage[a4paper, total={18cm, 27cm}]{geometry}
\usepackage{outlines}
\usepackage[sfdefault]{FiraSans}
\usepackage{hyperref}

\begin{document}
\maketitle

\noindent
\textit{\textbf{Requirements}: 1. Propose a paper topic to the professor that is sufficiently narrow for a 10-page paper.  2. Choose a topic that coveres a sub-field of Antarctic science.  3. Pay attention to details like how the data in the field is collected and analyzed, and where the research takes place.  4. Fully cover the sub-field.  5. Include references and at least one diagram.} \\
\begin{itemize}
\item\textbf{Paper topic}: The paper topic should be narrow enough to focus on details rather than broad statements about science reseach.  For example, \textit{climate science} might be narrowed to \textit{the impact of glacial melting on sea-level rise.}  Or, \textit{cosmology research} might be narrowed to \textit{contributions of South Pole Telescope} to cosmology in the last decade.
\item \textbf{Sub-field of Antarctic Science}: Example sub-fields have included: radio-glaciology, climate science, biology, and neutrino physics.
\item \textbf{Attention to detail}: The paper will receive a good grade if it sufficiently focuses on scientific details.  Specific details include how data is collected and processed, and how the experiments or expeditions are performed.  Examples of details that are less important but not insignificant are who is participating in the research (country of origin) and the history of their field.
\item \textbf{Full coverage}: When a sufficiently narrow topic is chosen, be sure to fully cover it, meaning do not ignore one group of scientific measurements or projects in favor of another.  Provide a level of detail for the whole sub-field instead of a laser focus on one measurement or outcome.
\item \textbf{References and diagrams}: Include at least one map or diagram, and at least 8-12 references.
\end{itemize}
\textit{\textbf{Example plan for research paper}:}
\begin{outline}[enumerate]
\1 Topic: neutrino physics, but specifically the contribution of the Askaryan Radio Array to neutrino physics.
\1 Sub-field: the sub-field happens to be neutrino physics performed in Antarctica.  Other projects include IceCube, ANITA, and ARIANNA.
\1 Attention to detail: the paper will cover various deployments of hardware, data collected, and how the hardware is meant to sense neutrinos.
\1 Full coverage: the paper will cover various deployments of hardware, data collected, and the implications of the results.  Other results from IceCube and ARIANNA will also be reviewed.
\1 References and diagrams: the references come from arXiv.org, and there is a map of where Askaryan Radio Array is located.
\end{outline}
\textbf{Grading}: The five criteria above are each assigned 20 percent of the grade.  Thus, a paper that reflects work in each category (1-5) will receive an excellent grade.  Good luck!
\end{document}
