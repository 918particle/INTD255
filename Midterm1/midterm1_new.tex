\title{INTD255, Safe Return Doubtful: Midterm 1}
\author{Dr. Jordan Hanson - Whittier College Dept. of Physics and Astronomy}
\date{\today}
\documentclass[10pt]{article}
\usepackage[margin=1.5cm]{geometry}
\usepackage{outlines}
\usepackage{graphicx}
\usepackage{amsmath}

\begin{document}
\twocolumn
\maketitle

\section{Early Antarctic and Oceanographic Exploration}

\begin{enumerate}
\item Who was Captain James Cook?  List several notable achievements of his three main expeditions. \\ \vspace{2cm}
\item What was the significance of the \textit{venus transit} on one of the Cook expeditions? What other, more secret, mission did Capt. Cook have in the Southern Ocean? \\ \vspace{2cm}
\item \textbf{Kepler's Laws:} If the units of orbital radius $r$ are AU, and the units of orbital period $T$ are years, then $T^2 = r^3$.  (a) If the orbital radius of Uranus is 19.22 AU, what is its orbita period in years? (b) If the orbital period of Mars is 1.88 years, what is its orbital radius? (c) What parameter in orbital mechanics was fixed by the observations of the venus transit in the late 18th Century by scientists who accompanied the Cook expedition? \\ \vspace{4cm}
\item List some of the achievements in early polar exploration attained on the following expeditions:
\begin{itemize}
\item \textit{The Belgica}
\item \textit{The Fram}, with Nansen
\item \textit{The Gjoa}
\item \textit{The Discovery}, with Scott
\end{itemize}
\item Who were the \textit{Dorset} and the \textit{Thule}?  How did they survive in their environment? \\ \vspace{1cm}
\item Who are the Chinook? Where did Capt. Cook approach their territory? \\ \vspace{1cm}
\item Discuss the risks and rewards of cultural exchange, in light of the writings of Barry Lopez in \textit{Horizon.}  As examples, consider the stories of Ranald MacDonald, Captain James Cook, Captain Amundsen and the Netsilik. \\ \vspace{5cm}
\end{enumerate}

\section{Survival Skills: Work, Energy, Food, and Physics}

\begin{enumerate}
\item How many kcal of energy is stored in 2 kg of pemmican? (Treat this as a fatty food, not a protein). \\ \vspace{2cm}
\item How many kg of wheat biscuits are required for 500 kcal of energy? (Treat this as mostly carbohydrates). \\ \vspace{2cm}
\item How many Joules of energy are required to pull 1000 kg across 5 km of snowy tundra, if the relevant coefficient of friction is 0.1? \\ \vspace{2cm}
\item Take your result from the previous exercise, and divide the energy among 10 sled dogs.  How much energy is required of each dog?  Now feed each dog that much pemmican.  How many kg of food, per dog, is required? \\ \vspace{2cm}
\item What food related health risk is associated with spending long durations at sea and in polar regions? \\ \vspace{1cm}
\end{enumerate}

\section{Navigation: Distance, Time, Speed, Longitude and Latitude}

\begin{enumerate}
\item How many nautical miles correspond to travelling 2.5 degrees directly South? \\ \vspace{2cm}
\item If we travel due North by 400 km, what is our change in latitude? \\ \vspace{2cm}
\item If we are travelling due West at a latitude of 60 deg North, what distance corresponds to a change of 1.5 degrees longitude? \\ \vspace{2cm}
\item If a ship sails East at 10 knots, how many nautical miles are travelled in 48 hours? \\ \vspace{2cm}
\end{enumerate}

\section{The British, The Norwegians, and Cultural Exchanges}

\begin{enumerate}
\item Having read the first part of the story of the race for the South Pole, describe the differences in style between the Norwegian/Scandanavian expeditions and the British ones. \textbf{Bonus:} Connect your ideas to indigenous cultural exchange, or our reading in \textit{Deep Survival.} \\ \vspace{2cm}
\item List five technologies for polar survival that the Norwegians learned from the \textit{Netsilik}. \\ \vspace{2cm}
\item What was the primary role of the Royal Geographic Society in British Antarctic exploration? \\ \vspace{2cm}
\item (a) How did the British travel and move gear in the polar regions, before motorized craft were developed? (b) How did this differ from the Norwegians? \\ \vspace{2cm}
\item In your view, what are the major risks to ships and explorers when exploring the polar regions in this period? \\ \vspace{2cm}
\item When American ambassadors first arrived in Japan, after a period of intense isolationism in Japan, they found the Imperial court already knew how to speak English?  How did this happen?  Who helped them to learn English, and of the desire of Western nations to trade? \\ \vspace{2cm}
\item Consider the following quote ``Whether the change facing a people comes on swiftly ... or slowly ... the responsibility of the wisdom keeper is to recognize the early signs of significant change, to look into the past, and locate, again, a through line to the future.'' Apply this idea to a story we have encountered in the course, or a problem facing our community today. \\ \vspace{5cm}
\end{enumerate}

\section{Exploration Achievements}

\begin{enumerate}
\item Discuss the cultural significance of Nansen's first crossing of Greenland to the nation of Norway. \\ \vspace{2cm}
\item Was Roald Amundsen the first leader to cross the North-West Passage? On whose progress did he build? \\ \vspace{2cm}
\item What were the highlights of Robert Falcon Scott's ``furthest South'' on the \textit{Discovery} expedition? \\ \vspace{2cm}
\item Scientifically, why was important for explorers in the early 20th century to locate the magnetic North and South poles? \\ \vspace{2cm}
\end{enumerate}

\section{Survival and Psychology}

\begin{enumerate}
\item Using the terminology found in \textit{Deep Survival}, what is the difference between a \textit{primary emotion}, and a \textit{secondary emotion?} \\ \vspace{1cm}
\item Using the terminology of emotional bookmarks, how are secondary emotions formed, and how do they protect you? \\ \vspace{3cm}
\end{enumerate}

\section{Reflections}

\begin{enumerate}
\item While on Skraeling Island, the write Barry Lopez interacts with a research team responsible for finding artifacts from Dorset, Thule, \textit{and Norse} cultures in Northern Canada.  What is the significance of these diverse finds, in your view? \\ \vspace{5cm}
\item While reflecting on the Pacific Ocean in Cape Foulweather, Lopez notices that you can never see the entire Pacific, because the Southern Ocean covers more than one hemisphere of area of our planet.  What an area of life or academic topic that you wish you understood, but might not ever fully understand? \\ \vspace{5cm}
\end{enumerate}

\end{document}
