\title{Study Guide for Midterm 1}
\author{Dr. Jordan Hanson - Whittier College Dept. of Physics and Astronomy}
\date{\today}
\documentclass[10pt]{article}
\usepackage[margin=1.5cm]{geometry}
\usepackage{outlines}
\usepackage{graphicx}
\usepackage{amsmath}

\begin{document}
\maketitle

\section{Memory Bank}

\begin{itemize}
\item $T^2 = R^3$ ... Kepler's 3rd Law, if $T$ is the orbital period in years and $R$ is the orbital radius in AU.
\item $W = F d$ ... Work (in Joules) is equal to force (in Newtons) times distance (in meters).
\item $f = \mu m g$ ... The force of friction if $f$ is in Newtons, $m$ is the mass in kilograms, and $g = 9.18$ m/s$^2$.  The number $\mu$ is called the coefficient of friction.
\item $W = \mu m g d$ ... Combining the above two formulas, we find the work in pulling a load against friction for some distance.
\item 1 kilocalorie, or 1 kcal, is equal to 4184 Joules.
\item The following conversions are useful: 1 gram of fat has 9 kcal of energy.  1 gram of protein has 4 kcal of energy.  1 gram of carbohydrate has 4 kcal of energy.
\item A distance \textit{vector} can be expressed as an amount of distance in a given direction.  We use the notation $\vec{x} = (a,b)$ to represent the amount of distance East ($a$), and the amount of distance North ($b$).
\item Vectors add like lists of numbers: $(a,b) + (x,y) = (a+x,b+y)$.
\end{itemize}

\section{The Planets}

\begin{enumerate}
\item Kepler's Third Law states that if the orbital period of a planet is given in \textit{years}, and the orbital radius is given in \textit{AU}, then 
\begin{equation}
T^2 = r^3
\end{equation}
For example, we can solve for the orbital radius like $r = T^{2/3}$.  If the period of Venus is $T=0.615$ years, then $r = (0.615)^{2/3} = 0.723$ AU.  Given the following orbital periods, solve for the orbital radii of the planets:
\begin{itemize}
\item Jupiter: 11.862 years
\item Saturn: 29.457 years
\item Pluto: 248 years
\end{itemize}
Solve for the following orbital periods:
\begin{itemize}
\item Mars: 1.524 AU
\item Uranus: 19.22 AU
\item Neptune: 30.11 AU
\end{itemize}
\end{enumerate}

\section{Work, Force, and Friction}

\section{Food Energy and Conversion}

\end{document}
