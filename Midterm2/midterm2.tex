\title{Study Guide for Midterm 2}
\author{Dr. Jordan Hanson - Whittier College Dept. of Physics and Astronomy}
\date{\today}
\documentclass[10pt]{article}
\usepackage[margin=1.5cm]{geometry}
\usepackage{outlines}
\usepackage{graphicx}
\usepackage{amsmath}

\begin{document}
\maketitle

\section{Memory Bank}

\begin{itemize}
\item $\vec{x} = a\hat{i} + b\hat{j}$ ... A distance vector that takes you a distance $a$ in the x-direction, and $b$ in the y-direction.
\item $|\vec{x}| = \sqrt{a^2 + b^2}$ ... The magnitude or length of the vector is given by Pythagorean theorem.
\item $a = |\vec{x}|\cos\theta$ ... The x-component of a vector, if $\theta$ is the angle made with the x-axis.
\item $b = |\vec{x}|\sin\theta$ ... The y-component of a vector, if $\theta$ is the angle made with the x-axis.
\item $\vec{x}_1 + \vec{x}_2 = (a_1 + a_2)\hat{i} + (b_1 + b_2)\hat{j}$ ... Adding two-vectors means adding the components.
\item $\vec{x}_1 - \vec{x}_2 = (a_1 - a_2)\hat{i} + (b_1 - b_2)\hat{j}$ ... Adding two-vectors means adding the components.
\item $W = m g h$ ... Energy required to climb a height $h$ (units: Joules).
\item $P = W/t$ ... Power is equal to work in Joules divided by time in seconds.  The units of power are called Watts.
\item $d = b/\tan(\theta)$ ... Suppose a distant object is a distance $d$ from an observer, and the observer walks a perpendicular baseline $b$ and measures the two compass headings pointing to the distant object: $\theta_1$ and $\theta_2$.  Let $\theta = \theta_1 - \theta_2$.  The distance to the object is the baseline divided by the tangent of the difference in the two compass headings.
\item $s = R\Delta\theta$ ... Using an angle of latitude: the distance $s$ is equal to the radius of the Earth times the change in latitude, $\Delta\theta$, where $\Delta\theta$ is measured in radians.  $\pi$ radians equals 180 degrees, $2\pi$ radians equals 360 degrees.
\item $s = \Delta\phi R \cos(\theta)$ ... Using an angle of longitude: the distance $s$ is equal to the radians of the Earth times the change in longitude $\Delta\phi$, times the cosine of the current latitude $\theta$.  Both angles should be in radians.
\item \textit{One nautical mile} is the distance corresponding to a change in latitude of 1/60th of one degree (a change of one ``minute'') at constant longitude.  It is equal to 1.852 kilometers.
\item The index of refraction $n$ of a material is the factor by which the speed of light is reduced, when light travels in that material.  If the speed of light is $c$, then the speed of light in a material is $v = c/n$.  The index of refraction for ice is 1.78, and the speed of light is $c \approx 3.0 \times 10^{8}$ m/s.
\item SCINI: acronymn for \textit{Submersible Capable of Under Ice Navigation and Imaging}.  A robotic probe capable of discovering life under Antarctic ice shelves.
\item SPT: acronymn for \textit{South Pole Telescope}.  A millimeter-wave telescope capable of detecting CMB photons.
\item CMB: Cosmic Microwave Background
\item $m = \rho V$ ... the mass $m$ of an object is equal to the density $\rho$ times the volume $V$.
\item $\Delta y = \frac{Ah}{l^2}\left(\frac{\rho_{\rm ice}}{\rho_{\rm H2O}}\right)$ ... The change in sea-level $\Delta y$ is given by the area $A$ and height $h$ of an iceberg dropped into the sea, if the sea has length and width $l$.  The density $\rho_{\rm ice}$ is that of ice, and the density $\rho_{\rm H20}$ is that of liquid water.
\item The radius of the Earth is 6371 km.
\item The value of $g$ is 9.81 m/s$^2$.
\item $\rho_{\rm H20} = 1000$ m/kg$^3$, and $\rho_{\rm ice} = 917$ m/kg$^3$.
\end{itemize}

\clearpage

\section{Navigation, 2}

\begin{enumerate}
\item Suppose a person walks 15 nautical miles at an angle 2 degrees West of North.  What is their position? \\ \vspace{1cm}
\item Suppose a ship sails from 30 degrees South latitude to 40 degrees South latitude. (a) How far have they traveled in nautical miles? (b) If they are sailing at 10 knots, how many hours will the journey take?  \\ \vspace{1.5cm}
\item Suppose a ship is sailing from London, UK to Newfoundland, Canada.  The voyage takes place at 50 degrees North latitude.  The ship sails from 0 degrees West longitude, and lands at 56 degrees West longitude.  How many kilometers did the ship travel? \\ \vspace{1.5cm}
\item Suppose we need to determine the distance to a distant landmark.  We walk a perpendicular baseline of 2.25 km, and take a compass heading.  The compass heading reads 281 degrees.  Returning to the original spot, we find that the heading has changed to 269 degrees.  How far away is the landmark?  \\ \vspace{1.5cm}
\item Suppose we scale a 15 meter wall in 150 seconds.  (a) How many Joules did we consume, if our mass is 50 kg? (b) What is our power consumption? \\ \vspace{1cm}
\end{enumerate}

\section{Biological Research in Antarctica}

\begin{enumerate}
\item \textbf{SCINI:} (a) Describe the scientific purpose of the SCINI and Deep SCINI modules. (b) Why did the module have to be constructed from special pressure resistant materials? (c) What was one major discovery of the SCINI project? \\ \vspace{2cm}
\item Name several pieces of evidence that Antarctic orcas are \textit{social} animals.  How does there hunting and other survival skills reveal that they can communicate and coordinate their behavior?  \\ \vspace{2cm}
\item Describe the interaction the photo-journalist had with the leopard seal in the TED talk.  What did we learn about the animal's behavior?
\end{enumerate}

\section{Cosmology Research in Antarctica}

\begin{enumerate}
\item Describe in your own words how the South Pole Telescope studies the big bang.  (a) How does the Cosmic Microwave Background serve as a piece of evidence for the big bang? \\ \vspace{2cm}
\end{enumerate}

\section{Climate Science in Antarctica}

\begin{enumerate}
\item Suppose we are trying to measure the ice thickness of a glacier, to determine if any ice has melted away relative to last year.  The index of refraction of ice is $n = 1.78$. (a) If a radio-wave takes 25.0 microseconds to go down through the ice, reflect off of the bedrock, and return to the surface, how thick is the ice? (b) Suppose we return to the same area, and the result has changed to 24.5 microseconds.  What is the new ice thickness?  (c) Subtract the answers in part (a) and (b) to obtain the change in thickness. \\ \vspace{2cm}
\item Suppose the glacier breaks away from land and enters a sea that is 500 km by 500 km.  The area of the iceberg is 100 km$^2$.  The height of the iceberg is measured to be 500 meters.  By how much does the sea level rise? \\ \vspace{2cm}
\item \textbf{Bonus Point:} In recent years, water has been spotted flowing from the surface of Greenland.  Suppose a river of water formed from melted ice, and reached the ocean.  If the trickle of water is measured to be the equivalent of 100 kg per day, to what volume of ice does this correspond? 
\end{enumerate}

\end{document}
