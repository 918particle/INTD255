\title{Study Guide for Midterm 2}
\author{Dr. Jordan Hanson - Whittier College Dept. of Physics and Astronomy}
\date{\today}
\documentclass[10pt]{article}
\usepackage[margin=1.5cm]{geometry}
\usepackage{outlines}
\usepackage{graphicx}
\usepackage{amsmath}

\begin{document}
\maketitle

\section{Memory Bank}

\begin{itemize}
\item $\vec{x} = a\hat{i} + b\hat{j}$ ... A distance vector that takes you a distance $a$ in the x-direction, and $b$ in the y-direction.
\item $|\vec{x}| = \sqrt{a^2 + b^2}$ ... The magnitude or length of the vector is given by Pythagorean theorem.
\item $a = |\vec{x}|\cos\theta$ ... The x-component of a vector, if $\theta$ is the angle made with the x-axis.
\item $b = |\vec{x}|\sin\theta$ ... The y-component of a vector, if $\theta$ is the angle made with the x-axis.
\item $\vec{x}_1 + \vec{x}_2 = (a_1 + a_2)\hat{i} + (b_1 + b_2)\hat{j}$ ... Adding two-vectors means adding the components.
\item $\vec{x}_1 - \vec{x}_2 = (a_1 - a_2)\hat{i} + (b_1 - b_2)\hat{j}$ ... Adding two-vectors means adding the components.
\item $W = m g h$ ... Energy required to climb a height $h$ (units: Joules).
\item $P = W/t$ ... Power is equal to work in Joules divided by time in seconds.  The units of power are called Watts.
\item $d = b/\tan(\theta)$ ... Suppose a distant object is a distance $d$ from an observer, and the observer walks a perpendicular baseline $b$ and measures the two compass headings pointing to the distant object: $\theta_1$ and $\theta_2$.  Let $\theta = \theta_1 - \theta_2$.  The distance to the object is the baseline divided by the tangent of the difference in the two compass headings.
\item $s = R\Delta\theta$ ... Using an angle of latitude: the distance $s$ is equal to the radius of the Earth times the change in latitude, $\Delta\theta$, where $\Delta\theta$ is measured in radians.  $\pi$ radians equals 180 degrees, $2\pi$ radians equals 360 degrees.
\item $s = \Delta\phi R \cos(\theta)$ ... Using an angle of longitude: the distance $s$ is equal to the radians of the Earth times the change in longitude $\Delta\phi$, times the cosine of the current latitude $\theta$.  Both angles should be in radians.
\item \textit{One nautical mile} is the distance corresponding to a change in latitude of 1/60th of one degree (a change of one ``minute'') at constant longitude.  It is equal to 1.852 kilometers.
\item The index of refraction $n$ of a material is the factor by which the speed of light is reduced, when light travels in that material.  If the speed of light is $c$, then the speed of light in a material is $v = c/n$.  The index of refraction for ice is 1.78, and the speed of light is $c \approx 3.0 \times 10^{8}$ m/s.
\item SCINI: acronymn for \textit{Submersible Capable of Under Ice Navigation and Imaging}.  A robotic probe capable of discovering life under Antarctic ice shelves.
\item SPT: acronymn for \textit{South Pole Telescope}.  A millimeter-wave telescope capable of detecting CMB photons.
\item CMB: Cosmic Microwave Background
\item $m = \rho V$ ... the mass $m$ of an object is equal to the density $\rho$ times the volume $V$.
\item $\Delta y = \frac{Ah}{l^2}\left(\frac{\rho_{\rm ice}}{\rho_{\rm H2O}}\right)$ ... The change in sea-level $\Delta y$ is given by the area $A$ and height $h$ of an iceberg dropped into the sea, if the sea has length and width $l$.  The density $\rho_{\rm ice}$ is that of ice, and the density $\rho_{\rm H20}$ is that of liquid water.
\item The radius of the Earth is 6371 km.
\item The value of $g$ is 9.81 m/s$^2$.
\item $\rho_{\rm H20} = 1000$ m/kg$^3$, and $\rho_{\rm ice} = 917$ m/kg$^3$.
\end{itemize}

\clearpage

\section{Navigation, 2}

\begin{enumerate}
\item Suppose a person walks 3 kilometers West, and 4 kilometers North.  How far are they from the origin? \\ \vspace{1cm}
\item Suppose a person walks 3 kilometers at an angle 30 degrees North of East.  What is their position?  That is, how far North are they, and how far East are they? \\ \vspace{1cm}
\item Suppose a ship sails from 30 degrees North latitude, to 30 degrees South latitude. (a)  How far have they traveled in nautical miles?  (b) What is this value in kilometers? (c) If they are sailing at 20 knots, how long will the journey take?  \\ \vspace{1.5cm}
\item Suppose we need to determine the distance to a distant mountain.  We walk a perpendicular baseline of 2 km, and take a compass heading.  The compass heading reads 92 degrees.  Returning to the original spot, we find that the heading has changed to 83 degrees.  How far away is the mountain?  \\ \vspace{1.5cm}
\item Suppose we know that the top of the mountain is about 3 km from the valley floor.  How many Joules of energy would be required for a 55 kg person to ascend this height?  \textit{Hint: convert the height to meters first.}  \\ \vspace{1cm}
\item If the climber did this ascension in three days, what is the average \textbf{power} consumption? \textit{Hint: you must find the number of seconds in three days.} \\ \vspace{1.5cm}
\item If someone runs at such a speed that their body consumes 500 Watts, and they run for 1 hour, what energy did they consume in Joules? \\ \vspace{2cm}
\end{enumerate}

\section{Biological Research in Antarctica}

\begin{enumerate}
\item \textbf{SCINI:} What made the discovery of fish and other life under the Ross Ice Shelf significant, since the discovery took place hundreds of kilometers from the ocean front? \\ \vspace{2cm}
\item Recall the hunting technique the Antarctic orcas used for seals in Graham Land (Antarctic Peninsula).  What does this technique tell us about the orcas' relationships with each other? \\ \vspace{2cm}

\end{enumerate}

\section{Cosmology Research in Antarctica}

\begin{enumerate}
\item What is the Cosmic Microwave background?  Why is this radiation in the \textit{microwave} bandwidth, rather than some other higher bandwidth? \\ \vspace{2cm}
\item How does measuring the CMB tell us about science of the Big Bang? \\ \vspace{2cm}
\end{enumerate}

\section{Climate Science in Antarctica}

\begin{enumerate}
\item Suppose we are trying to measure the ice thickness of a glacier, to determine if any ice has melted away relative to last year.  The index of refraction of ice is $n = 1.78$.  If a radio-wave takes 20 microseconds to go down through the ice, reflect off of the bedrock, and return to the surface, how thick is the ice? \\ \vspace{2cm}
\item Suppose the glacier breaks away from land and enters a sea that is 100 km by 100 km.  The area of the iceberg is 20 km$^2$.  The height of the iceberg is measured to be 400 meters.  By how much does the sea level rise? \\ \vspace{2cm}
\item What is the mass of the iceberg in the previous problem?
\end{enumerate}

\end{document}
