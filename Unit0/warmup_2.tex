\documentclass{article}
\usepackage{graphicx}
\usepackage[margin=1.5cm]{geometry}
\usepackage{amsmath}

\begin{document}
\twocolumn

\title{Warm Up: Unit 0: Energy, Work, Food, and Polar Travel}
\author{Prof. Jordan C. Hanson}

\maketitle

\section{Memory Bank}
\begin{itemize}
\item 1 calorie = 4.184 Joules
\item 1 kcal = 4184 Joules
\item Carbohydrates: 4 kcal/gram
\item Protein: 4 kcal/gram
\item Fat: 9 kcal/gram
\item 1 kilogram = 1000 grams
\item Work: $W = F d$, force times distance.  If force is in Newtons, and distance is in meters, work is in Joules.
\item Power: work divided by time.
\item 1 Watt = 1 Joule / 1 second
\item A Watt is a unit of \textit{power}, a Joule is a unit of \textit{energy.}
\item 1 Joule = 1 Newton $\times$ 1 meter
\item Force of friction: $f = \mu m g$, where $mu$ is the coefficient of friction, $m$ is the mass in kg, $g = 9.81$ m s$^{-2}$.
\item Working against friction: $W = \mu m g d$.  If $d$ is in meters, $mu$ is the coefficient of friction, $m$ is the mass in kg, and $g = 9.81$ m s$^{-2}$, the final units are Joules.
\end{itemize}

\section{The Depot Problem}

\begin{enumerate}
\item (a) How much energy in Joules does it take to pull 200 kg of food 10,000 meters, if the coefficient of friction for waxed wood and snow is 0.1? (b) Convert your response to kcal. (c)  If your average power output is 300 Watts (Joules per second), how many seconds will this activity require? \\ \vspace{6cm}
\item Suppose you reach the 10,000 meter mark, drop the 100 kg of food, and return to base (neglect the energy required to push yourself back).  Suppose you rest at base, then travel the 10,000 meters for free, and use the 200 kg of food to travel farther into the ice.  (a) If the food is worth 4 kcal/gram, and you have 200 kg, how many kcal do you have in the depot? (b) What is this energy in Joules? (b) Assuming the same conditions as before, how much farther can you travel?
\end{enumerate}

\end{document}
