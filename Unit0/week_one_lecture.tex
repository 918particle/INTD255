\documentclass{beamer}
\usetheme{metropolis}
\usepackage{graphicx}
\usepackage{subfig}
\title{Algebra-Based Physics-1: Mechanics (PHYS135A-01): Unit 0}
\date{\today}
\author{Jordan Hanson}
\institute{Whittier College Department of Physics and Astronomy}

\begin{document}
\maketitle

\begin{frame}{Course Introduction}
\begin{enumerate}
\item Professor Jordan Hanson
\item Contact: jhanson2@whittier.edu, SLC 212
\item Syllabus: Moodle
\item Office hours: Mondays, 16:30-17:30, and Tuesdays from 13:00-16:00 in SLC 212
\item Text: College Physics (openstax.org)
\item Homework: Assigned from the book Mondays, due the following Monday
\end{enumerate}
\end{frame}

\section{Opening Remarks - Exploration}

\begin{frame}{Opening Remarks - Exploration}
\textit{``We shall not cease from exploration \\
And the end of all our exploring \\
Will be to arrive where we started \\
And know the place for the first time. .''} \\  \vspace{1cm} \textbf{T.S. Eliot, \textit{Little Gidding}, from  \textit{Four Quartets}, 1936-42.}
\end{frame}

\section{Summary}

\begin{frame}{Unit 0 Summary}
\begin{enumerate}
\item Why explore?
\begin{itemize}
\item \textit{Little Gidding}, secs. IV and V
\item You \textit{can} explore, but why?
\item The other part...
\end{itemize}
\item Course syllabus
\item The books and journal
\item Warm up exercises: mileage
\item Force, energy, work, and friction
\item Unit conversions
\begin{itemize}
\item Currency
\item Energy
\end{itemize}
\item \textbf{The extent of the Solar System, part I}
\end{enumerate}
\end{frame}

\section{Why Explore?}

\begin{frame}{Why Explore?}
\textbf{Why explore?}
\begin{enumerate}
\item The world is a beautiful place
\item To learn things you do not know that you do not know
\item To learn about about yourself
\end{enumerate}
\end{frame}

\begin{frame}{Why Explore?}
\textbf{That other part...}
\begin{enumerate}
\item How to handle yourself in strange situations
\item How to use your brain to survive
\item \textit{What really matters}
\end{enumerate}
\end{frame}

\section{The Syllabus}

\begin{frame}{The syllabus}
\small
See moodle: \\ 
\url{https://cms.whittier.edu/course/view.php?id=23388}
\end{frame}

\section{Warm-up exercises}

\begin{frame}{Warm-ups}
\begin{enumerate}
\item How many gallons of gasoline will your vehicle hold?  (Or that of your family, friends).
\item What is the gas mileage on the highway?
\item How far can you go?
\end{enumerate}
\end{frame}

\section{Unit Conversions}

\begin{frame}{Unit Conversions}
We must learn how to deal with \textit{units.}
\begin{enumerate}
\item In 1900-1915, 1.0 USD equals 3.8 krone.
\item One Calorie, which equals 1 kilocalorie, is 4184 \textit{Joules}.
\end{enumerate}
\end{frame}

\begin{frame}{Unit Conversions}
\small
We must learn how to deal with \textit{units.}
\begin{enumerate}
\item \textbf{In 1900-1915, 1.0 USD equals 3.8 krone.}
\item One Calorie, which equals 1 kilocalorie, is 4184 \textit{Joules}.
\end{enumerate}
Question - The \textit{Primus stove} was invented in 1892 by Franz Wilhelm Lindqvist, from Sweden.  Suppose it cost 12.00 krone.  What did it cost in USD? \\ \vspace{0.5cm}
Question - A northern sled dog was required for sledging in the early 1900s.  Suppose one could be purchased for 50.00 USD.  What is that cost in krone? \\
\end{frame}

\begin{frame}{Unit Conversions}
\small
We must learn how to deal with \textit{units.}
\begin{enumerate}
\item In 1900-1915, 1.0 USD equals 3.8 krone.
\item \textbf{One Calorie, which equals 1 kilocalorie, is 4184 \textit{Joules}.}
\end{enumerate}
Question - An inactive person requires about 2000 Calories per day.  How many Joules does she require per day?  \\ \vspace{0.5cm}
Question - A typical source of protein contains 4.0 Calories per gram.  How many Joules are in 200 grams of protein? \\
\end{frame}

\section{Force, Work, and Energy}

\begin{frame}{Force, Work, and Energy}
\textbf{Force:} 
\begin{equation}
F = m a
\end{equation}
\begin{itemize}
\item F: Force, in Newtons (British unit: pounds or lbs.)
\item m: mass, in kilograms (British unit: stone)
\item a: acceleration, in meters per second squared (British unit: feet per second squared)
\end{itemize}
\textbf{Professor: work several examples.}
\end{frame}

\begin{frame}{Force, Work, and Energy}
\textbf{Work, or energy:} 
\begin{equation}
W = F d
\end{equation}
\begin{itemize}
\item F: Force, in Newtons (British unit: pounds or lbs.)
\item d: Distance, in meters (British unit: feet)
\end{itemize}
\textbf{Professor: work several examples.}
\end{frame}

\begin{frame}{Force, Work, and Energy}
\textbf{Force of friction:} 
\begin{equation}
f = \mu m g
\end{equation}
\begin{itemize}
\item f: Force of friction, in Newtons (British unit: pounds or lbs.)
\item $\mu$: Greek letter mu, unit-less constant ($\approx 0.01 - 0.1$)
\item g: acceleration downward due to gravity, or 9.81 meters per second squared.
\end{itemize}
\textbf{Professor: work several examples.}
\end{frame}

\section{Astronomy and Early Antarctic Exploration}

\begin{frame}{Astronomy and Early Antarctic Exploration}
James Cook and Charles Green, 1769
\begin{enumerate}
\item Kepler's Laws: $T^2 \propto r^3$
\item Kepler: Preduicted when Venus would \textit{transit} the Sun, from our perspective
\item What is an AU?
\item Sir Edmund Halley (1656 - 1742).  Halley's comet passed by in 1758 (16 years after he died).
\begin{itemize}
\item Devised a method for determining distance to the Sun.
\item Astronomers sent out all over the world (Baja California, Tahiti, etc.) to make the recordings.
\end{itemize}
\end{enumerate}
\end{frame}

\section{Conclusion}

\section{Summary}

\begin{frame}{Unit 0 Summary}
\begin{enumerate}
\item Why explore?
\begin{itemize}
\item \textit{Little Gidding}, secs. IV and V
\item You \textit{can} explore, but why?
\item The other part...
\end{itemize}
\item Course syllabus
\item The books and journal
\item Warm up exercises: mileage
\item Force, energy, work, and friction
\item Unit conversions
\begin{itemize}
\item Currency
\item Energy
\end{itemize}
\item \textbf{The extent of the Solar System, part I.}
\end{enumerate}
\end{frame}

\end{document}
