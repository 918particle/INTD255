\documentclass{article}
\usepackage{graphicx}
\usepackage[margin=1.5cm]{geometry}
\usepackage{csquotes}
\usepackage{url}

\begin{document}

\title{Calculating the Depth of Polar Ice using Radar}
\author{Prof. Jordan C. Hanson}

\maketitle

\section{Introduction}

In this activity, you will learn how researchers measure the depth of polar ice as it relates to studies of climate change.  The speed of light is normally a constant,

\begin{equation}
c = 3.0\times 10^{8}~~m/s
\end{equation}

That is, a wave of light (like a radio wave) travels 300 million meters per second in a vacuum.  However, in a material such as ice, the speed is

\begin{equation}
v = \frac{c}{n} \label{eq:n}
\end{equation}

In Eq. \ref{eq:n}, the variable $n$ is called the \textit{index of refraction.}  In general, it varies according to the frequency of the radio wave or wave of light.  At radio frequencies, the value is $n=1.78$.  Suppose a radio wave is shot downwards through ice, reflects off of the ocean below, and returns to the receiver.  If the ice thickness is $h$, then the total distance traveled by the radio wave will be $2h$.  In general, if $\Delta y$ is the total distance traveled, $v$ is the speed, and $\Delta t$ is the time, then

\begin{equation}
\Delta y = v \Delta t
\end{equation}

Speed is just the ratio of distance and time.  Let's assume that $\Delta y = 2h$, and $v = c/n$.  This implies that

\begin{equation}
2h = \frac{c}{n} \Delta t
\end{equation}

Suppose we measure $\Delta t$ and we assume $n = 1.78$.  The measured ice thickness would be given by

\begin{equation}
h = \frac{c\Delta t}{2n}
\end{equation}

\section{Examples}
Derive the ice thicknesses below:
\begin{enumerate}
\item Suppose we observe $\Delta t = 4.0$ $\mu$s.  One $\mu$s is equal to $10^{-6}$ seconds.  What is $h$?
\item Suppose we arrive at a place in the Antarctic where we observe $\Delta t = 6.1$ $\mu$s.  The next year, we observe $\Delta t = 6.02$ $\mu$s. By how much did the ice thickness decrease?
\end{enumerate}

\section{The Open Polar Server}

On the classroom PCs, go to the following website: \url{https://ops.cresis.ku.edu}.  Click on the tab in the upper left labeled \textbf{Antarctic}.  This should take you to a map of Antarctica, containing flight lines of aircraft collecting radar data over the ice sheets.

\begin{enumerate}
\item Click on one flight line and pull up a radar ``echogram.''  The y-axis represents $\Delta t$, and the x-axis represents distance or location.  The coler scale of the diagram represents the power $P$ of the radar returned signal.
\item On the right-hand y-axis of these diagrams we see the time delay in $\mu$s.  Identify the surface of the ice, and identify the bottom.  Estimate the times in $\mu$s.
\item Calculate the depth of the ice sheet along the flight line in five different places.
\end{enumerate}

\end{document}
