\documentclass{article}
\usepackage{graphicx}
\usepackage[margin=1.5cm]{geometry}
\usepackage{csquotes}

\begin{document}
\small
\twocolumn

\title{Wednesday Warm Up: \textit{Last Place on Earth}}
\author{Prof. Jordan C. Hanson}

\maketitle

\section{Chapter 24 - The Pole Seeker \\ Prepares}

\begin{enumerate}
\item Robert Falcon Scott ordered the \textit{Terra Nova} back to New Zealand for the winter after operations concluded in the summer.  After barely making it out of the pack ice, and the ocean freezing solid behind them, the ship made it to Lyttlelton, New Zealand.  There was a problem with the bilge pump, though, which nearly cost them the ship.  Why was it dangerous to have just one working pump?  What did Davies do to fix it?  (\textit{This type of situation is known as a single point of failure, or single POF}). \\ \vspace{2cm}
\item At ``Framheim'' Roald Amundsen and company establish a winter routine to enable good working morale through the long dark winter.  Describe some of the things they did to pass the time.  On what projects did they work?  \textbf{One example to get you started}: Roald Amundsen had the foresight to buy hickory wood from Pensacola \textit{a decade} prior to the expedition, to be used by Bjaaland for sledge-making.\\ \vspace{2cm}
\item What did the men at Framheim do with the navigational tables to avoid the single POF that occurred on Scott's prior expedition, when the only set of tables was lost just before the journey South? \\ \vspace{2cm}
\end{enumerate}

\section{Chapter 25 - Wintering at Cape Evans}

\begin{enumerate}
\item Robert Falcon Scott proposed leaving for the South Pole on November 3rd and taking 144 days to travel 1530 miles.  a) How many miles per day is this?  b) Based on what you know from prior units, does this seem reasonable if dogs are pulling the loads?  c) What methods did Scott choose to haul the gear, over the use of motor sledges and dogs? \\ \vspace{2cm}
\item One quality of leadership that was required of each captain was the ability to handle \textit{isolation.}  The author discusses how each leader deals with the paradox of being the person ``in charge,'' and therefore separate from the others, while at the same time being in close contact with the others all winter.  Compare and contrast the styles of Scott and Amundsen on this account.  \\ \vspace{2cm}
\end{enumerate}

\section{Chapter 26 - False Start}
\begin{enumerate}
\item In the days leading up to the official start of the journey to the Pole, Amundsen and his men kept pushing back the start date.  What was causing them to wait?  What was the effect on their nerves?  What finally lead to their clearange for departure? \\ \vspace{2cm}
\item The author describes the final start of Amundsen and company for the Pole with words like ``a marriage of civilization and primitive culture,'' and that ``it was to end the era of terrestrial exploration that began with the explosion of the human spirit during the Renaissance.''  Why does the author frame the adventure this way?  Do you agree with this characterization?  Why or why not?  \\ \vspace{2cm}
\end{enumerate}

\end{document}
