\documentclass{beamer}
\usetheme{metropolis}
\usepackage{graphicx}
\usepackage{tcolorbox}
\title{Safe Return Doubtful: Week 9}
\date{\today}
\author{Jordan Hanson}
\institute{Whittier College Department of Physics and Astronomy}

\begin{document}
\maketitle

\section{Summary}

\begin{frame}{Summary}
\textbf{Climate science}
\begin{enumerate}
\item \textbf{How do we know} adding ice to ocean will cause sea level rise? (The ice cubes in a glass concept).
\item How do we know \textbf{historical} temperature of ice?
\item How do we know the speed of ice flowing into ocean?
\end{enumerate}
\end{frame}

\section{Adding Ice to Ocean and Sea Level Rise}

\begin{frame}{Antarctic Ice and the Ocean - Examples}
The density of something is the mass divided by the volume:
\begin{equation}
\rho = \frac{m}{V}
\end{equation}
The units can be kilograms per meter cubed, or kg/m$^{3}$.  If we know the density of a material and the volume, we can know the mass of the object:
\begin{equation}
m = \rho V
\end{equation}
(Examples).
\end{frame}

\begin{frame}{Antarctic Ice and the Ocean - Examples}
The force of gravity, or weight (as we saw once before) is the mass of an object, times the gravitational \textit{acceleration}, called $g$:
\begin{equation}
w = mg
\end{equation}
The value of $g$ is 9.81 meters per second squared, or 9.81 m/s$^2$.  When an object floats, the buoyant force must balance the weight. When a mass in kilograms is multiplied by $g$, the units of force are \textit{Newtons.}  (Examples).
\end{frame}

\begin{frame}{Antarctic Ice and the Ocean - Examples}
\begin{tcolorbox}[colback=white,colframe=red!40!blue,title=The buoyant force]
\alert{The buoyant force is an upward force placed on an object suspended in a liquid that is equal in magnitude to the weight force of the displaced volume of liquid.}
\end{tcolorbox}
Example, suppose a sailboat displaces 45 cubic meters of water.  The weight force is equal to the mass of that water in the \textit{upwards direction.}  What's the mass of that water?  $m = \rho V$, so $m = 1000$ kg/m$^3 \times 45$ m$^3$, so 45,000 kg.  What is the \textit{weight} of that water?  $w = mg$, so $w = 45,000$ kg $\times 9.81$ m/s$^2$, or 441,450 Newtons (about 99,000 lbs).
\end{frame}

\begin{frame}{Antarctic Ice and the Ocean - Examples}
What is the \textit{weight} of that water?  $w = mg$, so $w = 45,000$ kg $\times 9.81$ m/s$^2$, or 441,450 Newtons (about 99,000 lbs).
\begin{figure}
\centering
\includegraphics[width=0.6\textwidth]{buoy.png}
\caption{\label{fig:buoy} A diagram of the buoyant force balancing the weight.}
\end{figure}
\end{frame}

\begin{frame}{Antarctic Ice and the Ocean - Examples}
\textbf{Questions for table discussion:}
\begin{itemize}
\item Given what you've seen, why does a ship made of metal float?  Isn't metal more dense than water?
\item What is the weight force in Newtons of a person with mass 55 kg?
\item What is the weight of an ice cube that is a cube of 10 cm on each side, and the density of ice is 918 kg/m$^3$?
\end{itemize}
\end{frame}

\begin{frame}{Antarctic Ice and the Ocean - Examples}
\textbf{Questions for table discussion:}
\begin{itemize}
\item Given what you've seen, why does a ship made of metal float?  Isn't metal more dense than water?
\item What is the weight force in Newtons of a person with mass 55 kg?
\item What is the weight of an ice cube that is a cube of 10 cm on each side, and the density of ice is 917 kg/m$^3$?
\end{itemize}
\end{frame}

\begin{frame}{Antarctic Ice and Ocean Level Rise}
\begin{figure}
\centering
\includegraphics[width=0.7\textwidth]{ice.jpg}
\caption{\label{fig:ice} An iceberg is mostly below the water.  Can we determine what fraction of the iceberg is below the water, and therefore the displaced water?}
\end{figure}
\end{frame}

\begin{frame}{Antarctic Ice and Ocean Level Rise}
Suppose a block of ice of height $V$ and horizontal area $A$ is floating in the ocean.  The volume is
\begin{equation}
V = Ah
\end{equation}
Let the density of ice be $\rho_{\rm ice}$, and let the density of water be $\rho_{\rm w}$.  The weight of the iceberg is
\begin{align}
w &= mg \\
w &= \rho_{\rm ice} V g \\
w &= \rho_{\rm ice} A h g
\end{align}
\textbf{Table discussion:} What is the \textit{weight} of an iceberg that has an area of 1 km by 1 km on top, a height of 1 km?
\end{frame}

\begin{frame}{Antarctic Ice and Ocean Level Rise}
\small
Continuing, the weight of the iceberg is
\begin{equation}
w_i = \rho_{\rm ice} A h g
\end{equation}
How much water does the iceberg displace?  Let the height \textit{above} the water be $x_1$ and the depth below the water be $x_2$, such that $x_1 + x_2 = h$.  The displaced water volume is
\begin{equation}
V_d = A x_2
\end{equation}
The displaced water mass is $m_w = \rho_{w} V_d$, so
\begin{equation}
m_w = \rho_{w} A x_2
\end{equation}
The weight of the displaced water just requires us to multiply by $g$, so
\begin{equation}
W_w = \rho_{w} A x_2 g
\end{equation}
\end{frame}

\begin{frame}{Antarctic Ice and Ocean Level Rise}
Thus, we have the weight of the displaced water, $\rho_{w} A x_2 g$, and the weight of the iceberg, $\rho_{\rm ice} A h g$. According to the buoyant force principle, these two weights are \textbf{equal.}  Thus, 
\begin{equation}
\rho_{w} A x_2 g = \rho_{\rm ice} A h g
\end{equation}
Given that $h = x_1 + x_2$, let's simplify this...(board).
\begin{equation}
\frac{x_1}{x_2} = \frac{\rho_{\rm w} - \rho_{\rm ice}}{\rho_{\rm ice}} \label{eq:ice1}
\end{equation}
Equation \ref{eq:ice1} is the fraction of the iceberg that peaks above the surface.
\end{frame}

\begin{frame}{Antarctic Ice and Ocean Level Rise}
\begin{equation}
\frac{x_1}{x_2} = \frac{\rho_{\rm w} - \rho_{\rm ice}}{\rho_{\rm ice}} \label{eq:ice2}
\end{equation}
Equation \ref{eq:ice2} is the fraction of the iceberg that peaks above the surface.  Given that you can see $\approx 10$ percent of an iceberg, and that $90$ percent is beneath the surface, we know how much water is displaced. \\
\begin{equation}
V_{wd} = A x_1 \left(\frac{\rho_{\rm ice}}{\rho_{\rm w} - \rho_{\rm ice}}\right)
\end{equation}
What is the water level rise if we toss an iceberg into a giant bucket of water? \\ \vspace{0.5cm}
Suppose the giant bucket has a length and a width of $l$, so an area of $l^2$.  What is the water level rise $\Delta y$?
\end{frame}

\begin{frame}{Antarctic Ice and Ocean Level Rise}
Here's how to solve for it:
\begin{align}
l \times l \times \Delta y &= V_{wd} \\
\Delta y &= \frac{Ah}{l^2}\left(\frac{\rho_{\rm ice}}{\rho_{\rm w}}\right)
\end{align}
What is the sea level rise if $l = 1$ km, $h = 1$ km, and $A = 1$ km$^2$?  Recall that ice density is 917 kg/m$^3$, and water density is 1000 kg/m$^3$.
\end{frame}

\begin{frame}{Antarctic Ice and Ocean Level Rise}
So that was an example of how adding ice to the ocean causes sea level rise.  Here is a short video summary of research into Antarctic and Greenland contributions to sea level rise: \\ 
\url{https://youtu.be/YRe1ymYR45k} \\ \vspace{1cm}
And here is how certain people view the science: \\
\url{https://youtu.be/lPgZfhnCAdI}
\end{frame}

\begin{frame}{Summary}
\textbf{Climate science}
\begin{enumerate}
\item \textbf{How do we know} adding ice to ocean will cause sea level rise? (The ice cubes in a glass concept).
\item (Next time): How do we know \textbf{historical} temperature of ice?
\item (Next time): How do we know the speed of ice flowing into ocean?
\end{enumerate}
\end{frame}


\end{document}
