\documentclass{article}
\usepackage{graphicx}
\usepackage[margin=1.5cm]{geometry}
\usepackage{csquotes}

\begin{document}

\title{Thursday Reading Assessment: Chapters 33-35 of \textit{Last Place on Earth}, Chapters 1 and 3 of \textit{News at the Ends of the Earth}.}
\author{Prof. Jordan C. Hanson}

\maketitle

\section{Chapter 33 - The Ultimate Defeat}

\begin{enumerate}
\item The author describes how \textit{stress} is actually a drain on Vitamin C.  If this is true, it suggests that Captain Scott and his party were dangerously low on Vitamin C for two reasons.  First, they could not gain it from their depots.  Second, what reserves were stored in their bodies were sapped by the stress of being lost.  How does this situation connect to the reading from \textit{Deep Survival} by Laurence Gonzales? \\ \vspace{2cm}
\end{enumerate}

\section{Chapter 34 - The Birth of a Legend}

\begin{enumerate}
\item Once the crew of the \textit{Terra Nova} admitted to themselves that Captain Scott's party was not coming back, they frantically loaded the ship and sailed up to Hut Point, erecting a cross in honor of the Captain and his men.  ``The inscription, chosen by Cherry-Garrard, was the same line from Tennyson's \textit{Ulysses}, `To strive, to seek, to find, and not to yield,' that Nansen chose when praising Amundsen's attainment of the North-West Passage in London six years before.''  That happens to be the modern location of McMurdo Station - the home of the United States Antarctic Program.  Why do you think Scott's men chose this quote?  What were some of the achievements of Scott in polar exploration?  How have these led to advances in the modern era?  (\textit{Hint: as an example, consider the idea of the motor sledge.} \\ \vspace{2cm}
\end{enumerate}

\section{Chapter 35 - The Last Adventure}

\begin{enumerate}
\item There are several interesting details in this final chapter.  Choose one of the following and explain its significance in your own words: (a) Captain Amundsen had to pause exploration because the Great War (WWI) finally broke out.  What did he do with the situation?  (b) English schoolchildren were taught that Captain Scott discovered the South Pole.  Why was this happening? (c) Roald Amundsen later learned to fly aircraft, and obtained Norway's first civilian pilot's license.  What was the result in the end? \\ \vspace{2cm}
\end{enumerate}

\section{Introduction and Chapter 1 of The News at the Ends of the Earth}

\begin{enumerate}
\item The book begins with a quote from George Simpson, one of Captain Scott's men, from the \textit{South Polar Times}.  He appears to have written a science fiction short story featuring people from Sirius 8, a starship, who have discovered climate records left by the British Antarctic Expedition, which predict global warming (this was written in 1911).  How does this broaden your notion of people's climate awareness, and capacity for imagining space exploration? \\ \vspace{2cm}
\item In Chapter 1, Hester Blum writes about the technical challenge of actually printing newspapers in Arctic and Antarctic regions.  Yet, she points out that that in almost every Arctic and Antarctic expedition of the late 19th and early 20th centuries, newspapers and magazines were written by the explorers.  She goes on to point out that ``as Tocqueville describes in \textit{Democracy in America}: `A newspaper ... always represents an association, the members of which are its readers.  That association can be more or less well-defined, more or less restricted, and more or less numerous... .'''  Who were the audiences of Scott and Amundsen as the men wrote journals, magazies and newspapers?  In what senses were their writings \textit{local}, and \textit{national}? \\ \vspace{2cm}
\end{enumerate}

\section{Chapter 3 of The News at the Ends of the Earth}
\begin{enumerate}
\item In this chapter, we learn some of the strange beliefs that modern society held about the Antarctic before major exploration took place.  Authors like the American Edgar Allen Poe and Canadian James de Mille portrayed the Antarctic as "temporate" or "teeming with life."  Other contemporaries believe the inner core of the Earth could be accessed via the Antarctic!  The explorers' groups consisted of a mixture of scientific experts and men who could barely read and write, and they often chose to hold classes and share writing about meteorology and ideas about whether the climate could change during the long night.  Discuss the significance of these activities.  What does it reveal about the character of those leading the expeditions (excluding Amundsen's, which did precisely none of this except private journals)?  
\end{enumerate}

\end{document}
