\title{Syllabus for Safe Return Doubtful: \textit{History and Current Status of Modern Science in Antarctica} (INTD-255)}
\author{Dr. Jordan Hanson - Whittier College Dept. of Physics and Astronomy}
\date{\today}
\documentclass[10pt]{article}
\usepackage[a4paper, total={18cm, 27cm}]{geometry}
\usepackage{outlines}
\usepackage[sfdefault]{FiraSans}
\usepackage{hyperref}
\begin{document}
\maketitle

\begin{abstract}
The history and current status of cutting edge science on the Antarctic continent will be presented. The course will begin with the heroic and perilous adventures of Robert Falcon Scott, Ernest Shackleton, and Roald Amundsen in the early 20th century. Often described as a playground for scientific research, Antarctica has proven to be a treasure trove for breakthrough scientific discoveries and engineering breakthroughs over the past 100 years. The course will cover the initial discoveries and navigation of the Antarctic continent, and qualitative and quantitative details regarding landmark achievements in physics, astrophysics, geophysics, biology and climate science.
\end{abstract}
\noindent
\textit{\textbf{Pre-requisites}: none.} \\
\textit{\textbf{Course credits, Liberal Arts Categorization}: 3 Credits, CON2} \\
\textit{\textbf{Regular course hours}: Tuesda and Thursday from 11:00 - 12:20 in SLC 228} \\
\textit{\textbf{Instructor contact information}: jhanson2@whittier.edu, tel. 562.907.5130} \\
\textit{\textbf{Office hours}: } \\
\textit{\textbf{Attendance/Absence}: Students needing to reschedule midterms and exams should notify the professor a reasonable time beforehand. Further attendance issues are left to the discretion of the instructor}.\\ 
\textit{\textbf{Late work policy}: Late work is generally not accepted, but is left to the discretion of the instructor.} \\
\textit{\textbf{Reading}:}
\begin{enumerate}
\item ``The Last Place on Earth: Scott and Amundsen's Race to the South Pole,'' by Roland Huntford.  Modern Library Paperback Ed., 1999.  This will be the main text of the course, which covers one of the definitive human exploration stories of the last hundred years.
\item ``Horizon,'' by Barry Lopez.  Alfred A. Knopf, Penguin Random House LLC, 2019.  A recent work from a fantastic travel writer who provides a detailed picture of modern Antarctic expedition and research.
\item ``The News at the Ends of the Earth: \textit{The Print Culture of Polar Exploration},'' by Hester Blum. Duke University Press, 2019.  This text covers newspapers and other collected writings of the polar explorers, and reveals what their observations tells us about their perceptions of the environment, including climate change.
\item \textbf{Please purchase a journal, in which specific weekly journal assignments will be written.}  Journal assignments are due weekly, and should be $\approx 10$ pages on the given topic, and include the date and title of the assignment.
\item ``Deep Survival: \textit{Who Lives, Who Dies, and Why},'' by Laurence Gonzalez.  Norton, 2017.  This delightful book is optional to purchase, and it describes the philosophy of those who live through survival situations.  Scans of relevant chapters will be provided.
\end{enumerate}
\textit{\textbf{Grading}: There will be two midterms, each covering both the scientific and exploration reading, and requiring the student to solve problems based on the reading and class material. Each midterm is worth 20\% of the final grade. There will be one 10-page (single-spaced) essay covering a current scientific field of study in Antarctica, worth 20\% of the final grade. Weekly journal assignments will be compiled into a longer essay (15-20 pages), and will be due at the end of the semester and worth 20\% of the final grade.  Finally, in-class participation in activities and discussion will be worth 20\% of the final grade.  In-class activities will include reading quizzes, group discussions, group activities and excursions.} \\
\textit{\textbf{Grade Settings}: $<60\%$ = F, $\geq 60\%, <70\%$ = D, $\geq 70\%, <80\%$ = C, $\geq 80\%, <90\%$ = B, $\geq 90\%, <100\%$ = A.  Pluses and minuses: 0-3\% minus, 3\%-6\% straight, 6\%-10\% plus (e.g. 79\% = C+, 91\% = A-)} \\
\textit{\textbf{Homework}: \textbf{There will be 50-60 pages of reading per week} throughout the semester, and students are strongly encouraged to be disciplined in completing the readings.} \\
\textit{\textbf{ADA Statement on Disability Services}: The Americans with Disabilities Act (ADA) is a federal anti-discrimination statute that provides comprehensive civil rights protection for persons with disabilities. Among other things, this legislation requires that all students with disabilities be guaranteed a learning environment that provides for reasonable accommodation of their disabilities. If you believe you have a disability requiring an accommodation, please contact Disability Services: disabilityservices@whittier.edu, tel. 562.907.4825.} \\
\textit{\textbf{Academic Honesty Policy}: \url{http://www.whittier.edu/academics/academichonesty}}
\clearpage
\textit{\textbf{Course Objectives}:}
\begin{itemize}
\item To practice written and oral expression of scientifically technical ideas.
\item To practice journal writing and the summary of journal writing into a complete work.
\item To solve word problems pertaining to physics, astronomy, navigation, nutrition, and climate science.
\item To experience the planning and execution of logistically challenging expeditions and operations. 
\end{itemize}
\textit{\textbf{Course Outline}:}
\begin{outline}[enumerate]
\1 \textbf{Week 1}
\2 Reading: Last Place on Earth, ch. 1-4. Beginning of part I.
\2 Reading quiz: none
\2 Basic survival calculations, warm up.  Fuel costs and distances.
\2 Activity: Lecture on Two Expeditions to Moore's Bay.
\2 Weekly journal assignment: ``Survival Situations.'' Write about a time you or someone you know was in a survival situation and explain what they did to respond.
\1\textbf{Week 2}
\2 Reading: Last Place on Earth, ch. 5-8.
\2 Reading quiz: Last Place on Earth, ch. 1-4.
\2 Activity: Lecture on neutrino physics: why Antarctica is important for physics
\2 Weekly journal assignment: ``Different.''  Write about a time you encountered a place, or space that was wholly different than your home.  Alternatively, write about a time you encountered a person or people who was or were wholly different than yourself or your family.  Record what you learned from this experience.
\1 \textbf{Week 3}
\2 Reading: Last Place on Earth, ch. 9-12.
\1 \textbf{Week 4}
\2 Reading: Last Place on Earth, ch. 13-16. (Ch. 17 optional). End of part I.
\1 \textbf{Week 5}
\2 Reading: Last Place on Earth, ch. 18-20. Beginning of part II.
\1 \textbf{Week 6}
\2 Reading: Last Place on Earth, ch. 21-23.
\1 \textbf{Week 7}
\2 Reading: Last Place on Earth, ch. 24-26.
\1 \textbf{Week 8}
\2 Reading: Last Place on Earth, ch. 27-29.
\1 \textbf{Week 9}
\2 Reading: Last Place on Earth, ch. 30-32.
\1 \textbf{Week 10}
\2 Reading: Last Place on Earth, ch. 33-35. End of part II.
\1 \textbf{Week 11}
\2 Reading: 
\1 \textbf{Week 12}
\2 Reading: 
\1 \textbf{Week 13}
\2 Reading: 
\end{outline}
\end{document}
