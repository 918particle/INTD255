\title{Syllabus for Safe Return Doubtful: \textit{History and Current Status of Modern Science in Antarctica} (INTD-255)}
\author{Dr. Jordan Hanson - Whittier College Dept. of Physics and Astronomy}
\date{\today}
\documentclass[10pt]{article}
\usepackage[a4paper, total={18cm, 27cm}]{geometry}
\usepackage{outlines}
\usepackage{hyperref}
\begin{document}
\maketitle

\begin{abstract}
The history and current status of cutting edge science on the Antarctic continent will be presented. The course includes the heroic and perilous adventures of Robert Falcon Scott, Ernest Shackleton, and Roald Amundsen in the early 20th century. Often described as a playground for scientific research, Antarctica has proven to be a treasure trove for breakthrough scientific discoveries and engineering breakthroughs over the past 100 years. The course will cover the initial discoveries and navigation of the Antarctic continent, and qualitative and quantitative details regarding landmark achievements in physics, astrophysics, geophysics, biology and climate science.
\end{abstract}
\noindent
\textit{\textbf{Pre-requisites}: none.} \\
\textit{\textbf{Course credits, Liberal Arts Categorization}: 3 Credits, CON2.} \\
\textit{\textbf{Regular course hours and location}: Monday and Wednesday, 11:00 - 12:20, SLC 228.} \\
\textit{\textbf{Instructor contact information}: Discord: 918particle, Email: jhanson2@whittier.edu, Office: SLC 212, Book appointments: \url{https://fgucmvjkylvmgqfsco.10to8.com}.} \\
\textit{\textbf{Office hours}: Use booking service link above to schedule meetings.} \\
\textit{\textbf{Attendance/Absence}: Students needing to reschedule midterms must notify the professor a few days in advance.} \\ 
\textit{\textbf{Late work policy}: Late work is generally not accepted, but is left to the discretion of the instructor.} \\
\textit{\textbf{Reading}: Each week, students will be expected to complete all of the reading.  There will be reading quizzes to ensure students receive credit for staying current with the reading.  Following the reading, students will reflect on the reading through a journal exercise.  Students will also receive credit for completing the journal exercise.}
\begin{enumerate}
\item ``The Last Place on Earth: Scott and Amundsen's Race to the South Pole,'' by Roland Huntford.  Modern Library Paperback Ed., 1999.  This will be the main text of the course, which covers one of the definitive human exploration stories of the last hundred years.
\item ``Horizon,'' by Barry Lopez.  Alfred A. Knopf, Penguin Random House LLC, 2019.  A recent work from a fantastic travel writer who provides a detailed picture of modern Antarctic expedition and research.
\item ``Deep Survival,'' by Laurence Gonzales.  W. W. Norton and Company, 2017.  A fascinating book about people survive, or do not survive, in dangerous situations.
\item \textbf{Please purchase a journal, in which specific weekly journal assignments will be written.}  Journal assignments are due weekly, and should be $\approx 10$ pages on the given topic, and include the date and title of the assignment.
\end{enumerate}
\textit{\textbf{Grading}: There will be two midterms, each covering both the scientific and exploration reading, and requiring the student to solve problems based on the reading and class material. Each midterm is worth 20\% of the final grade. There will be one 10-page essay covering a current scientific field of study in Antarctica, worth 20\% of the final grade. Weekly journal assignments will be compiled into a longer essay due at the end of the semester, and will be worth 20\% of the final grade.  Finally, in-class participation in activities and discussion will be worth 20\% of the final grade.  In-class activities include reading quizzes, group discussions, group activities, and excursions.} \\
\textit{\textbf{Grade Settings}: $<60\%$ = F, $\geq 60\%, <70\%$ = D, $\geq 70\%, <80\%$ = C, $\geq 80\%, <90\%$ = B, $\geq 90\%, <100\%$ = A.  Pluses and minuses: 0-3\% minus, 3\%-6\% straight, 6\%-10\% plus (e.g. 79\% = C+, 91\% = A-)} \\
\textit{\textbf{Homework}: \textbf{There will be 50-60 pages of reading per week} throughout the semester, and students are strongly encouraged to be disciplined in completing the readings.} \\
\textit{\textbf{ADA Statement on Disability Services}: Whittier College is committed to make learning experiences as accessible as possible. If you experience physical or academic barriers due to a disability, you are encouraged to contact Student Disability Services (SDS) to discuss options. To learn more about academic accommodations, email disabilityservices@whittier.edu, call (562) 907-4825, or go to SDS which is located on the ground floor of Wardman Library.} \\
\textit{\textbf{Academic Honesty Policy}: \url{http://www.whittier.edu/academics/academichonesty}} \\
\clearpage
\textit{\textbf{Course Objectives}:}
\begin{itemize}
\item To practice written and oral expression of scientifically technical ideas
\item To review and understand the significance of polar research
\item To practice journal writing and the summary of journal writing into a complete work
\item To solve word problems pertaining to physics, astronomy, navigation, nutrition, and climate science
\item To experience the planning and execution of logistically challenging expeditions and operations
\end{itemize}
\textit{\textbf{Course Outline}:}
\begin{outline}[enumerate]
\1 \textbf{Week 0}
\2 Reading: \textit{Horizon}, ``Looking for a Ship.''
\2 Reading: \textit{The Last Place on Earth}, ch. 1-4.
\2 Reading quiz: \textit{Horizon} and \textit{The Last Place on Earth}
\2 Basic survival calculations, warm up.  Fuel costs and distances, unit conversions.
\2 Activity: Exploration of the size of the Solar System, exploration of the Northwest Passage
\2 Weekly journal assignment: ``Survival Situations.'' Write about a time you or someone you know was in a survival situation and explain what they did to respond. \\
\1\textbf{Week 1}
\2 Reading: \textit{Horizon}, ``Cape Foulweather.''
\2 Reading: \textit{The Last Place on Earth}, ch. 5-7.
\2 Reading quiz: \textit{Horizon} and \textit{The Last Place on Earth}
\2 Basic survival calculations, warm up.  Calories and energy, the depot problem.  Work, energy, and friction.
\2 Activity: Lecture on Two Expeditions to Moore's Bay.
\2 Weekly journal assignment: ``Different.''  Write about a time you encountered a place, or space that was wholly different than your home.  Record what you learned from this experience. \\
\1 \textbf{Week 2}
\2 Reading: \textit{Horizon}, ``Skraeling Island.''
\2 Reading: \textit{The Last Place on Earth}, ch. 8.
\2 Reading quiz: \textit{Horizon} and \textit{The Last Place on Earth}
\2 Activity: Lecture the importance of Antarctica for neutrino physics
\2 Activity: Native technologies and Vitamin C.  Diet composition, clothing, and methods of transport.
\2 Weekly journal assignment: ``Diet.''  Record exactly what you eat, each day, for one week.  How many calories do you estimate you consume each day? \\
\1 \textbf{Week 3}
\2 Reading: \textit{The Last Place on Earth}, ch. 9-13.
\2 Reading quiz: \textit{The Last Place on Earth}
\2 Activity: Cost and shipping of equipment, instruments, and rations.
\2 Activity: Lecture on glaciology, radio-glaciology, CReSIS, connections to physics and climate science.
\2 Weekly journal assignment: ``Pushing yourself.''  Write about a time that you pushed yourself near your physical limitations.  Good examples are training for sports in high school or staying in shape.  Alternatively, write about a time you pushed yourself to your mental limitations.  Good examples are studying for a really important exam or dealing with a mentally taxing situation.  How did you handle it?  With hindsight, what would you have done differently? \\
\clearpage
\1 \textbf{Week 4}
\2 Reading: \textit{The Last Place on Earth}, ch. 14-17.
\2 Reading: \textit{Deep Survival}, prologue and ch. 1-2.
\2 Reading quiz: none.
\2 Activity: Lecture on glaciology, radio-glaciology, CReSIS, connections to physics and climate science.
\2 Activity: Review for Midterm 1.
\2 Weekly journal assignment: Reflect on journal entries from Weeks 0-3. \\
\1 \textbf{Week 5, February 21st, 2025 - Midterm 1.}
\2 Review reading and journal entries
\2 Comprised of short essay questions regarding reading and calculations surrounding the scientific concepts.
\1 \textbf{Week 6}
\2 Reading: \textit{The Last Place on Earth}, ch. 18-20.
\2 Reading: \textit{Deep Survival}, ch. 3-4
\2 Reading quiz: \textit{The Last Place on Earth} and \textit{Deep Survival}
\2 Activity: Leadership exercise, designing an excursion to nearby trails. Trail choices: Hellman Park Trail, Turnbull Canyon Trail, or Arroyo Pescadero Trail. Designing of procedures, styles of leadership.
\2 Activity: Short lecture on sled dogs, animal behavior, and navigation with compasses.
\2 Weekly journal assignment: ``Over the Seas.''  Choose a place on the map to which you would like to travel via the sea.  Chart a course, and calculate \textit{with as much detail possible} the distances, headings, and time required to make the journey.  Who would come with you?  Include details about necessary quantities of rations, and how you would avoid problems like vitamin C and B deficiencies. \\
\1 \textbf{Week 7}
\2 Reading: \textit{The Last Place on Earth}, ch. 21-23.
\2 Reading: \textit{Deep Survival}, ch. 5-6
\2 Reading quiz: \textit{The Last Place on Earth} and \textit{Deep Survival}
\2 Activity: Finish excursion planning to nearby trails.  Practice triangulation with compasses.
\2 Activity: Lecture on biology of the Ross Sea, and under the Ross Ice Shelf. SCINI and similar projects.
\2 Weekly journal assignment: ``Small.'' Describe a time when you felt \textit{small}.  Not necessarily a bad thing, feeling small can occur when you encounter a place more vast than your normal experience.  Other examples can include encountering an animal much larger than a human, or encountering a holy site or site of cultural significance that relates one's experience to that of history or the divine.  What was it like to be in the presence of something that reveals your idea of \textit{self} to be small in comparison to the world? \\
\1 \textbf{Week 8}
\2 Reading: \textit{The Last Place on Earth}, ch. 24-26.
\2 Reading \textit{Deep Survival}, ch. 7-8.
\2 Reading quiz: \textit{The Last Place on Earth} and \textit{Deep Survival}
\2 Activity: Short lecture on depots, snow types (variable friction), Amundsen's strategy.
\2 \textbf{Activity: Execute the planned excursion to trails.}
\2 Weekly journal assignment: no weekly journal assignment. \\
\1 \textbf{Week 9}
\2 Reading: \textit{The Last Place on Earth}, ch. 27-29.
\2 Reading \textit{Deep Survival}, ch. 9-10.
\2 Reading quiz: \textit{The Last Place on Earth} and \textit{Deep Survival}
\2 Activity: ``Ice maze.''  Solve the problem of laying depots and achieving transport to navigate from one wilderness location to the other.  This activity will draw on both physics and nutrition.
\2 Activity: Lecture on astrophysics, astroparticle physics, and polar regions.  The South Pole Telescope, BICEP 1-2, IceCube, IceCube Gen2, ARIANNA, ARA, ANITA I-IV, PUEO, and RNO-G.
\2 Weekly journal assignment: ``The Goal of My Life.'' What is the goal of your life?  Write about the one, over-arching goal of your life's work and how you chose it and have decided to accomplish it.  If you have never thought of this, treat this as a reflection of how you \textit{should} spend your life.  What principles or values emerge that guide you in your decisions? \\
\1 \textbf{Week 10, March 28th, 2025 - Midterm 2.}
\2 Review reading and journal entries
\2 Comprised of short essay questions regarding reading and calculations surrounding the scientific concepts.
\1 \textbf{Week 11}
\2 Reading: \textit{The Last Place on Earth}, ch. 30-32
\2 Reading: \textit{Deep Survival}, ch. 11-12.
\2 Reading quiz: \textit{The Last Place on Earth} and \textit{Deep Survival}
\2 Activity: Lecture on climate science research in Antarctica, connections to radio-glaciology (papers of K. Matsuoka, J. MacGregor, E. Rignot and others).
\2 Activity: Tour of machine shop and laboratory, summary of drone research.
\2 Activity: Survival, error propagation, emotional memory, the backfire effect, and bending the map.
\2 Weekly journal assignment: ``Lost.'' Write about a time you were lost.  Where were you when you started?  What information convinced you that you were lost, and how did you find your way back?  For extra credit, find a building or facility (on campus), and get lost in it.  Once there, ask yourself: in which direction is my house/dormitory?  Include in your writing how you managed to stay oriented despite being in a new place. \\
\1 \textbf{Week 12}
\2 Reading: \textit{The Last Place on Earth}, ch. 33-35.
\2 Reading: \textit{Deep Survival}, ch. 13-14.
\2 Reading quiz: \textit{The Last Place on Earth} and \textit{Deep Survival}
\2 Activity: Lecture on The Dry Valleys and Mars Exploration.
\2 Activity: Lecture on engineering challenges in the Antarctic.  Ideas from the PolarTech conference.
\2 Weekly journal assignment: ``Letters Home.'' Think back to a time you were away from home, or imagine you were in a different country.  Write a journal entry meant to preserve your observations so that your family and friends could experience that place as you did.  Include imporant details such as the date, specific locations, and names of places.  \textit{Details matter.} \\
\1 \textbf{Week 13}
\2 Reading: \textit{Deep Survival}, ch. 15.
\2 Reading: \textit{Horizon}, ``Graves Nunataks to Port Famine Road.''
\2 Reading quiz: \textit{Deep Survival} and \textit{Horizon}
\2 Activity: The West Point Leadership Lecture.
\2 Activity: Group reflection of what makes a good leader.
\2 Weekly journal assignment: ``Leadership.'' Think of a time you were asked to lead.  What were your responsibilities?  How did you address them?  Did the event go as planned?  Assess your leadership performance and discuss what you might have done differently. \\
\1 \textbf{Week 14}
\2 Reading: no reading, in order to focus on essays.
\2 Activity: Lecture on Geophysics research in Antarctica.
\2 Weekly journal assignment: ``The Climate Future.'' People have long studied the question of anthropogenic effects on the climate of the Earth.  Reflect on how a changing climate will affect your life.  What purchases, activities, and behaviors will you be forced to change?  What effects of climate change will not have any effect on your life?  Name several examples of how you can help the situation in the community. \\
\2 \textbf{End of Week 14 - scientific essay is due.} Students will submit an essay (2500 - 5000 words) that summarizes a scientific sub-field or activity currently taking place or that took place in a specific region of Antarctica.  Examples include the physiology of scurvy, under-ice biological research, or the activities of chemists studying climate change in the dry valleys near McMurdo.  The essay will include references, and at least two diagrams or maps.  The essays will be graded on four elements: \textit{inclusion of detail}, \textit{scientific accuracy}, \textit{coverage of the scientific sub-field} (i.e. including many sources), \textit{inclusion of references and diagrams}. \\
\1 \textbf{Week 15}
\2 Reading: no reading, in order to focus on essays.
\2 \textit{Final thoughts and reflections.}
\2 \textbf{End of Week 15 - class journal summary is due.} Students will submit a summary of what they have learned from keeping the class journal.  This writing assignment is by nature more open-ended than the scientific essay.  However, it must incorporate elements from each journal entry, and should be at least 2500 words.  This piece of writing should have a clear thesis, a beginning, and a conclusion.
\end{outline}
\end{document}